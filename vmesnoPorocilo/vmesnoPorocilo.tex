\documentclass{article}

\usepackage{marvosym}
\usepackage{amsmath}
\usepackage{amsfonts}
\usepackage{babel}

\renewcommand\refname{Vir}


\author{Gašper Terglav}
\date{5. April 2024}
\title{Najkrajša pot z odstranljivimi ovirami -- kratko poročilo}


\begin{document}


\maketitle

\section*{Delo do sedaj}

Predvsem sem bral članek in ga poskušal razumeti. Potem sem še na hitro napisal Dijkstra algoritem. Posledično je večina poročila samo predstavitev problema, nakoncu pa je še nek okviren načrt.

\section*{Predstavitev problema}

Problem je posplošitev klasične verzije, v kateri se lahko oviram samo izogibamo. V $\mathbb{R}^2$ imamo dve točki s in t iščemo najkrajšo evklidsko pot med njima. Ovire v ravnini so konveksni večkotniki, vsak od njih ima ceno $c_i > 0$. Če imamo na voljo $C$ ''denarja'', katere ovire se splača odstraniti, da dosežemo najkrajšo pot? V resničem življenju lahko npr. načrtovalci mest spremenijo cestno omrežje, da dosežejo boljšo pretočnost in tako odstranijo ovire za neko ceno. Še en primer omenjen v članku so skladišča v katerih delajo roboti. Kako spremeniti postavitev ovir v skladišču, da se roboti hitreje premikajo okoli? Problem je NP-težek (v članku je opisana redukcija problema PARTITION na naš problem), mi pa se bomo osredotočili na polinomske algoritme z napako.

% \subsection*{NP-težkost}

% Naš problem je NP-težek, kar lahko pokažemo z redukcijo na problem PARTITION, ki je NP-poln. Pri tem problemu imamo množico $A = \{a_1, a_2, ..., a_n\}$ pozitivnih celih števil. Vprašanje je, ali lakho $A$ razdelimo v množici $A_1$ in $A_2$, tako da je $W(A_1) = W(A_2) = \frac{1}{2}W(A)$, kjer je $W(S)$ vsota vseh elementov v $S$. Redukcijo dosežemo tako, da med točki $s$ in $t$ postavimo $n$ ovir, tako da je cena odstranitve enaka dolžini, za katero se pot podaljša, če ovire ne odstranimo. Če obstaja pot med s in t dolžine $\frac{1}{2}W(A) + \lVert s - t \rVert$ , kjer imamo $\frac{1}{2}W(A)$ proračuna za odstranitev ovir, potem smo našli razdelitev $A$ na dva enaka dela. Ovire, ki smo jih odstranili damo v eno možico, ostale pa v drugo. Konstrukcija je podrobneje opisana v članku. 

\subsection*{Algoritem v polinomskem času z napako $(1+\epsilon)$ v ceni}

Začnimo s pomembnima opazkama o obnašanju najkrajše poti. Če bo šla naša pot skozi oviro, bo zaradi konveksnosti sekala samo dve stranici večkotnika. Smer bo pot spremenila samo v ogliščih večkotnikov. Tako bomo lahko problem prevedli na iskanje najkrajše poti v grafu, katerega vozlišča so vsa oglišča ovir ter točki $s$ in $t$. Označimo ta graf z $G = (V, E)$, kjer $(u,v) \in E$, če je vsota cen vseh ovir, ki jih prečka daljica $uv$ manjša ali enaka od našega proračuna $C$. Vsaka povezava v grafu ima dva parametra, dolžino in ceno. Ta graf bi radi modificirali tako, da bi na novem grafu samo pognali Dijkstra algoritem in dobili rešitev, zato hočemo odstraniti parameter cene iz povezav. 
Naj bo  $K = min(\frac{C}{min_i \ c_i}, h)$, kjer je $h$ število ovir, $c_i$ pa njihove cene ter $\sum c_i = C$. Delimo vse cene z $K/C$, da je naš proračun za ovire na tako enak $K$. Potem naredimo $ \lceil{\frac{2K}{\epsilon}}\rceil + 1$  kopij vsakega vozličša v grafu $G$ in jih označimo z $v_0, v_{\epsilon/2}, v_{\epsilon}..., v_K$. Za vsak rob $(u, v) \in E$ s ceno $c$ in za vsak $0 \le i \le \lceil{\frac{2K}{\epsilon}}\rceil $, dodamo rob $(u_{i \epsilon/2}, v_{j \epsilon/2})$, kjer je $j\le\lceil{\frac{2K}{\epsilon}}\rceil$ največje celo število da velja $j\frac{\epsilon}{2}
 \le i \frac{\epsilon}{2} + c$. Dolžina povezav je enaka kot v originalnem grafu. Tako smo ceno zakodirali v vozlišča. Če se nahajamo v vozlišču $v_i$ to pomeni, da smo za odstranitev ovir do sedaj plačali $i$ približno denarja. Na koncu dodamo še novi vozlišči $s$ in $t$, ter vsakega od njiju povežemo z vsemi njunimi kopijami. Te povezave imajo dolžino $0$. Ker imajo povezave samo še dolžino, lahko sedaj za iskanje najkrajše poti uporabimo Dijkstro. Novi graf ima $O(|V|K/\epsilon)$ vozlišč in $O(|E|K/\epsilon)$ robov. Ker je Dijkstra najbolj časovno zahtevna stvar, je potem časovna zahtevnost celotnega algoritma enaka $O(\frac{K}{\epsilon}(|E| + |V|\log\frac{|V|}{\epsilon}) )$. Če je $n$ število vseh oglišč ovir, je to v najboljšem primeru $\Omega(\frac{n^3}{\epsilon})$.

\subsection*{Hitrejši algoritem}

Ta algoritem je bolj zapleten in doseže večjo hitrost z zmanjšanjem povezav v grafu $G$. V zameno pa je tukaj lahko napaka $(1+\epsilon)$ tudi pri dolžini poti. Ozaničmo nov graf z manj povezavami s $H = (X,\Gamma)$. Zanimivo je da ima $H$ več vozlišč kot $G$, bolj natančno $|X|, |\Gamma| = O(n \log n)$. Ko enkrat skonstruiramo $H$ pa lahko spet uporabimo idejo iz prejšnjega algoritma in naredimo $O(1/\epsilon)$ kopij ter uporabimo Dijkstro. Ta algoritem ima časovno zahtevnost $O(\frac{nh}{\epsilon^2}\log n \log \frac{n}{\epsilon})$ kar je približno za potenco $n$ manj kot prejšnji algoritem.


\section*{Okviren načrt dela}

Najprej nameravam v pythonu (ker imam z njim največ izkušenj) implementirati prvi algoritem za iskanje poti. To iskanje bi tudi grafično prikazal, za enkrat je ideja s pomočjo $matplotlib$, mogoče namesto tega spletni vmesnik. Naredil bom nekaj testnih primerov in opazoval kako se algoritem obnaša za različne $\epsilon$. Kasneje bom poskusil z implementacijo drugega algoritma in podatkovne strukture, ki omogoča hitrejše poizvedbe.


\begin{thebibliography}{99}

    \bibitem{PNS} Pankaj K. Agarwal, Neeraj Kumar, Stavros Sintos, and Subhash Suri. \emph{Computing Shortest Paths in the Plane with Removable Obstacles}. In 16th Scandinavian Symposium and Workshops on Algorithm Theory (SWAT 2018). Leibniz International Proceedings in Informatics (LIPIcs), Volume 101, pp. 5:1-5:15, Schloss Dagstuhl – Leibniz-Zentrum für Informatik (2018)

\end{thebibliography}
\end{document}